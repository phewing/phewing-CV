% Author: Phillip Hampton Ewing, Jr.
% Email: phewing@alum.mit.edu
% Website: www.phillipewing.com

% --------------------------------------------------------------------------- %

\documentclass[letterpaper, oneside, 10pt]{article}

\usepackage{multicol}
\usepackage{hologo}
\usepackage{phewing-cv}

% --------------------------------------------------------------------------- %

\renewcommand{\FirstName}{Phillip}
\renewcommand{\MiddleName}{Hampton}
\renewcommand{\LastName}{Ewing}
\renewcommand{\NameSuffix}{Jr.}
\renewcommand{\PhoneNumber}{\input{phone-number.txt}}
\renewcommand{\Email}{phewing@alum.mit.edu}
\renewcommand{\Website}{www.phillipewing.com}

\setcounter{secnumdepth}{0}
\renewcommand{\PageCount}{6}

% =========================================================================== %

\begin{document}
\StartCV%

\section*{Summary} % -------------------------------------------------------- %

Cross-disciplinary designer, researcher, and educator exploring software and
devices for creating environments that (1) improve users’ health, (2) promote
wellbeing, and (3) enhance cognitive performance, by applying knowledge and
experience from:

\Topic{Design}\DotSep{0.25em} \LN{3}+ years of work at
internationally-recognized design firms and labs; experience and knowledge
across multiple domains, including architecture/interiors, graphics,
experiments and surveys, software, as well as electronics.

\Topic{Research}\DotSep{0.25em} Leadership and publication of multiple
projects ranging from architectural robotics to workplace wellness, with
specialized expertise in light and human circadian rhythms.

\Topic{Programming}\DotSep{0.25em} Development of simulation tools and
applications for building performance analysis using Python, JavaScript,
C\#, R (among other languages) and version management tools (Git,
Mercurial).

\Topic{Fabrication}\DotSep{0.25em} Project experience with a broad
range of digital prototyping workflows, including printed circuit board
\A{(PCB)} layout and fabrication.

\Topic{Teaching}\DotSep{0.25em} Lectures, workshops, and
curriculum for architecture programs at multiple universities.




\section*{Education} % ------------------------------------------------------ %
\AdjSectSpace

\subsection*{%
  Massachusetts Institute of Technology (MIT)%
  \DotSep{0.25em} Cambridge, MA%
}
  \subsubsection*{%
    Master of Science In Architecture Studies (SMArchS),%
    \ Design and Computation\DotSep{0.25em} 2015%
  }
    \TopicEntry{Thesis}\ProperPub{%
      Interactive Phototherapy: Integrating Photomedicine into%
      Interactive Architecture%
    }\\%
    \TopicEntry{Cumulative GPA}\LN{4.8 (of 5.0)}%


\subsection*{Auburn University\DotSep{0.25em} Auburn, AL}
  \subsubsection*{Bachelor of Architecture (BArch)\DotSep{0.25em} 2012}
  \vspace{-6.5pt}
  \subsubsection*{Bachelor of Interior Architecture (BIArch)\DotSep{0.25em} 2012}
    \TopicEntry{Honors}\Foreign{Magna Cum Laude}\\
    \TopicEntry{Cumulative GPA}\LN{3.6 (of 4.0)}


\subsection*{%
  University of Alabama in Huntsville (UAH)%
  \DotSep{0.25em} Huntsville, AL%
}
  \subsubsection*{(General Studies)\DotSep{0.25em} 2006-2007}
    \TopicEntry{Cumulative GPA}\LN{4.0 (of 4.0)}



\section*{Skills} % --------------------------------------------------------- %
\AdjSectSpace%

\setlength{\multicolsep}{10pt}
\begin{multicols}{2}
{%
  \RaggedRight%
  {\jostmedium 3D/CAD/CAE:}%
    \begin{itemize*}[%
      label=\relax, labelwidth=0pt, itemjoin=\space\char"00B7%
    ]%
      \item \LN{3ds Max}%
      \item Auto\A{CAD}%
      \item Blender%
      \item Digital Project%
      \item Dynamo%
      \item \A{EAGLE}%
      \item Grasshopper%
      \item Inventor%
      \item Maya%
      \item Revit%
      \item \LN{Rhino 3D}%
      \item SketchUp%
      \item SolidWorks%
    \end{itemize*}


  {\jostmedium Analog Fabrication:}%
    \begin{itemize*}[%
      label=\relax, labelwidth=0pt, itemjoin=\space\char"00B7%
    ]%
      \item Composites molding/casting%
      \item Microelectronics production%
      \item Model-making/prototyping%
      \item Plastics molding/casting%
    \end{itemize*}


  {\jostmedium Building Simulation:}%
    \begin{itemize*}[%
      label=\relax, labelwidth=0pt, itemjoin=\space\char"00B7%
    ]%
      \item \A{ALFA}%
      \item \A{DIVA}%
      \item Ecotect%
      \item EnergyPlus%
      \item \A{IES VE}%
      \item OpenStudio%
      \item Radiance%
    \end{itemize*}


  {\jostmedium CAM:}%
    \begin{itemize*}[%
      label=\relax, labelwidth=0pt, itemjoin=\space\char"00B7%
    ]%
      \item \A{ABB} Robot Studio%
      \item \A{CNC}js%
      \item Fab Modules \A{(MIT)}%
      \item \LN{PartWorks 2D}%
      \item Surf\A{CAM}%
    \end{itemize*}


  {\jostmedium Digital Fabrication:}%
    \begin{itemize*}[%
      label=\relax, labelwidth=0pt, itemjoin=\space\char"00B7%
    ]%
      \item \LN{3D printing}%
      \item \A{CNC} milling%
      \item Laser cutting%
      \item Vinyl cutting%
      \item Waterjet cutting%
    \end{itemize*}


  {\jostmedium Libraries/Frameworks:}%
    \begin{itemize*}[%
      label=\relax, labelwidth=0pt, itemjoin=\space\char"00B7%
    ]%
      \item Arduino \A{SDK} (C/C++)%
      \item Google Maps \A{API} (JavaScript)%
      \item Grasshopper \A{SDK} (C\#, IronPython)%
      \item OpenCV (C++, Python)%
      \item TensorFlow w/Keras (Python)%
      \item Three.js (JavaScript)%
    \end{itemize*}


  {\jostmedium Markup:}%
    \begin{itemize*}[%
      label=\relax, labelwidth=0pt, itemjoin=\space\char"00B7%
    ]%
      \item \A{CSS}%
      \item \A{HTML}%
      \item Markdown%
    \end{itemize*}


  {\jostmedium Media Creation:}%
    \begin{itemize*}[%
      label=\relax, labelwidth=0pt, itemjoin=\space\char"00B7%
    ]%
      \item Dreamweaver%
      \item Final Cut Pro%
      \item Illustrator%
      \item InDesign%
      \item Inkscape%
      \item Photoshop%
      \item Premeire%
      \item Scribus%
    \end{itemize*}


  {\jostmedium Programming/Scripting:}%
    \begin{itemize*}[%
      label=\relax, labelwidth=0pt, itemjoin=\space\char"00B7%
    ]%
      \item C/C++%
      \item C\#%
      \item Java%
      \item JavaScript%
      \item \LaTeX\ /\ \hologo{BibTeX}%
      \item Python%
      \item R%
    \end{itemize*}

  {\jostmedium (Miscellaneous):}%
    \begin{itemize*}[%
      label=\relax, labelwidth=0pt, itemjoin=\space\char"00B7%
    ]%
      \item bash%
      \item CMake%
      \item Git%
      \item \A{HDF5}%
      \item libradtran%
      \item Mercurial%
      \item \A{MS-DOS}%
      \item Visual Paradigm (\A{UML}/SysML)%
    \end{itemize*}
  }
\end{multicols}

\section*{Recent Experience} % ---------------------------------------------- %

\hfill
\vspace{-24pt}

\subsection*{Perkins+Will\DotSep{0.25em} Atlanta, GA}
\subsubsection*{Researcher, Neuroarchitecture Lab\DotSep{0.25em} May 2016 -- Present}

\Topic{Neuroarchitecture Lab}%
\DotSep{0.25em} \textit{(formerly Human Experience Lab)}%


\textit{Circadian Light Tracker}\DotSep{0.25em}
Won and executed a firmwide ``innovation incubator'' research grant to
design, fabricate, and program (Arduino \A{SDK}, C/C++) a
wearable circadian light tracker equipped with 13 + 1 (spectral +
photopic) light sensors, and presented project results to the Perkins+Will
Atlanta office. (Publication and further development options currently
being actively explored.)

\textit{Circadian Light Analysis}\DotSep{0.25em}
Unilaterally implemented a novel, proof-of-concept analysis tool
(`pChroma') for simulating light exposure levels for building occupants
according to five spectral photosensitivity functions (i.e., `pentachromic'
functions) recommended by the latest circadian photobiologic research.
Successfully led publication and presentation of a conference paper
detailing research \& development work for the 2017 \A{IBPSA}
(International Building Performance Simulation Association) Building
Simulation conference (see Ewing et al. 2017).

\textit{Thermal Comfort}\DotSep{0.25em}
Reporting directly to the \A{CEO} (Phil Harrison,
\A{FAIA}), led a team of four Neuroarchitecture Lab researchers
in designing and deploying an online survey assessing employee thermal
comfort in the Perkins+Will Atlanta office, as well as evaluating and
reporting survey results.

\textit{Psychoacoustics}\DotSep{0.25em}
Worked directly with the Neuroarchitecture Lab director (Dr. Eve
Edelstein) in deploying survey protocols for assessing speech
intelligibility, ambient noise levels, and other psychoacoustic properties
of spaces using industrial-grade sound level and noise dose meters, sound
analysis software, and building occupant hearing tests (adapted from
\A{HINT} and \A{SIN} hearing tests). Instructed
Neuroarchitecture lab colleagues in conducting measurements and surveys.


\Topic{Energy Lab}

\textit{%
  Simulation Platform for Energy Efficient Design \A{(SPEED)}%
}\DotSep{0.25em}%
Collaborated as one of the primary software developers towards the
successful beta deployment of \A{SPEED}: an online application
for Perkins+Will designers to (1) easily create valid parametric models of
common building geometries and envelope constructions, (2) run up to
thousands of energy performance simulations per day on \A{AWS}
machines running OpenStudio Server, and (3) review 
data visualizations and analytics of simulation results. (Firmwide release
pending.)

\Topic{Design Process Lab}

\textit{Design Space Construction \A{(DSC)}} \DotSep{0.25em}
Collaborated with Process Lab researchers on the development of \A{DSC}, a
tool and workflow for optimizing building design performance based on both
quantitative (e.g., \A{EUI}) and qualitative (e.g., view quality) design
performance indicators. Further collaborated on curricula, and other
assorted materials explaining the \A{(DSC)} methodology for a workshop at
the 2016 \A{ACADIA} (Association for Computer-Aided Design in Architecture)
conference, 2017 and 2018 presentations at Autodesk University, and a 2018
journal article published in ITCon (see Haymaker et al. 2018).

\textit{Space Plan Generator \A{(SPG)}} \DotSep{0.25em}
Evaluated and advised algorithmic improvements to software developers for
development of software tools (Dynamo C\# \A{SDK}) of
\A{SPG}, a set of Dynamo plugins for automated generation of
floor plan design options based specification of program entities and
adjacencies via spreadsheet input. Assisted in multiple internal workshops
for instructing Perkins+Will healthcare planners on how to use
\A{SPG} for generating floor plan options for client projects.

\textit{Client Projects} \DotSep{0.25em}
Applied the \A{DSC} methodology for advising multiple
Perkins+Will project teams on building design performance and
potential improvement strategies.

\textit{Teaching Initiatives} \DotSep{0.25em}
Contributed curricula and documentation for workshops on design space
construction taught to graduate-level architecture students at the
University of Washington (Seattle), as well as 4\textsuperscript{th}-year
architecture undergraduate students at Auburn University.


\hfill
\vspace{-14.5pt}

\Topic{Building Technology Lab}

\textit{Thermoplastic Robotics} \DotSep{0.25em}
Collaborated with Building Technology Lab (Tech Lab) and Autodesk
\A{BUILD} space researchers in exploring designer-centric
workflows for digital fabrication of thermoformed plastic structural
systems using multiple, synchronized industrial robotic arms controlled
via Grasshopper, Dynamo, and \A{ABB} Robot Studio.


\Topic{General}

\textit{``Stand Up to Work'' Study}\DotSep{0.25em}
Assisted in executing a 12-month study of employees in the Perkins+Will
Atlanta office, funded by the American Society of Interior Designers
\A{(ASID)} and conducted in collaboration with the Center for
Active Design, the Icahn School of Medicine at Mount Sinai, and Steelcase,
on standing-desk usage in the workplace. Verified that `on-the-ground'
conditions complied with study protocols and communicated user survey
instructions and next steps to study participants and local office
management.

\textit{Gadget Lab}\DotSep{0.25em}
Operated, maintained, and debugged a variety of environmental testing and
measurement equipment as part of an in-house ‘gadget lab,’ including:
sound level and noise dose meters, handheld spectrometers, a distributed
ambient temperature and relative humidity sensor kit (Pointelist), a
smart flooring system, and other devices. Instructed research colleagues
on equipment operation and advised on acquisition of additional
measurement devices.

\textit{(Miscellaneous)}\DotSep{0.25em}
Developed Python scripts to automate tasks for literature search results
analysis, light \& sound measurement data presentation slides generation.
Implemented quasi-Monte Carlo sampling for design of experiments (DoE) via
interface with R, and a spectral daylight analytical model (Hosek-Wilkie)
as Grasshopper components using the C\# \A{SDK}. Implemented
scripted automation of other administrative tasks using Python.


\subsection*{Changing Places Group, MIT Media Laboratory\DotSep{0.25em} Cambridge, MA}

\subsubsection*{Project Assistant\DotSep{0.25em} April -- May 2015}

Collaborated with a team of graduate researchers and undergraduate
students on architectural design for the third iteration of CityHome: a
modular, ``mix-and-match'' kit of robotic actuators and sensors for
creating transformable, intelligent living spaces.

Produced design development drawings, digital models (Rhino, SketchUp),
and renderings (\kern-0.5pt\LN{3ds Max}) to refine development and
presentation of a 350 ft\textsuperscript{2} built apartment prototype.

Presented CityHome prototype (alongside project team) for an international
audience of Media Lab corporate sponsor representatives and press during
the Spring 2015 \A{MIT} Media Lab Members’ Week.


\subsubsection*{Research Assistant\DotSep{0.25em} April -- May 2015}

Led architectural design in collaboration with a team of graduate
researchers and undergraduate students for the second iteration of
CityHome: a robotic furniture module for converting a 200
ft\textsuperscript{2} micro-apartment space into a ``smart home'' with the
functionality of a space ``three times its size''; incorporates gesture,
voice and touch interfaces for user control; and allows for further
customization via downloadable ``apps''.

Produced schematic designs, \LN{3D}\ models (Rhino, Grasshopper, Autodesk
Inventor), construction drawings, hardware mockups, and G-code files
(\A{CNC} router, laser cutter, waterjet cutter) for the robotic
wall unit exterior finishes, primary structure, and portions of the
mechanical hardware.

Coordinated architectural designs with the mechatronics lead for the
robotic systems.

Completed design and construction (alongside project team) of functional
robotic wall unit and full-scale 200 ft\textsuperscript{2} demonstration
space in less than three months for live demonstrations during the Spring
2014 \A{MIT} Media Lab Members’ Week.

\suppresstrue


\section*{Teaching Experience} % -------------------------------------------- %

\hfill
\vspace{-25pt}


\subsection*{Auburn University\DotSep{0.25em} Auburn, AL}


\subsubsection*{Guest Lecturer\DotSep{0.25em} August -- December 2018}

Developed and presented three lectures, three workshops, and assignments on
building performance analysis using the Design Space Construction workflow
(Grasshopper, EnergyPlus, Radiance) to the entire 4\Ord{th}-year architecture
student body in ``\A{ARCH} \LN{2220}: Environmental Controls II” (Instructor:
Prof. David Hinson, \A{FAIA}). Reviewed and graded student assignments.


\subsubsection*{Guest Lecturer + Critic\DotSep{0.25em} June – August 2018}

Presented two lectures on acoustics and lighting analysis to 4th-year
undergraduate students in ``\A{ARIA} \LN{4030}: Interior Architecture
Thesis'' studio (Instructors: Profs. Kevin Moore, Matt Hall, Lida Sease).
Conducted two desk critique sessions and participated as a juror for midterm
and final reviews.


\subsubsection*{Guest Critic\DotSep{0.25em} April 2018}

Invited and attended as a design juror to review undergraduate thesis student
projects for ``\A{ARCH} \LN{5020}: Thesis Studio'' (Instructors: Profs. Justin
Miller, Randal Vaughn).


\subsection*{University of Washington\DotSep{0.25em} Seattle, WA}

\subsubsection*{Guest Lecturer\DotSep{0.25em} October 2017}

Co-presented a lecture and on circadian light analysis (Grasshopper, Radiance,
Python) for graduate students in the seminar course ``\A{ARCH}\LN{526}:
Topics in High Performance Buildings'' (Instructors: Prof. Chris Meek, Devin
Kleiner, Heather Burpee) as part of a workshop on the Design Space
Construction \A{(DSC)} workflow for building performance analysis. Co-wrote
the documentation and instructions on running \A{(DSC)} workflow.


\subsubsection*{Guest Lecturer\DotSep{0.25em} October 2017}

Co-presented a lecture on circadian light analysis (Grasshopper, Radiance,
Python) and conducted desk critiques of projects for graduate architecture
students enrolled in ``\A{ARCH} \LN{504}: Graduate Design Studio''
(Instructors: Prof. Chris Meek, David Kleiner).


\subsection*{%
  Singapore University of Technology and Design (SUTD)%
  \DotSep{0.25em} Singapore, SG\DotSep{0.25em}\textit{and}%
}
\subsection*{%
  Massachusetts Institute of Technology (MIT)\DotSep{0.25em} Cambridge, MA%
}

\subsubsection*{Research Assistant\DotSep{0.25em} June -- August 2013}

Developed a curriculum for an introductory class on Building Information
Modeling \A{(BIM)} for the Singapore University of Technology and Design
(\A{SUTD}; in collaboration as a `sister' university of \A{MIT}), emphasizing
\A{BIM} as part of a larger conceptual framework incorporating digital
fabrication workflows and ``digital twins'' for building operation and
monitoring. Produced semester-long course syllabus, content plan, and the
first five weeks of: lecture outlines, step-by-step lab session instructions,
digital models (Revit), and student assignments.


\setlength{\pagetotal}{\pagetotal - 12em}

\subsection*{Auburn University\DotSep{0.25em} Auburn, AL}

\subsubsection*{Guest Lecturer\DotSep{0.25em} August 2011 -- May 2012}

Developed teaching materials and instructed workshops on Rhino, Grasshopper,
and design application interoperability for undergraduate and graduate
architecture students. Advised on the development of additional digital media
workshops.

\subsubsection*{Teaching Assistant\DotSep{0.25em} August -- December 2010}

Led lab sessions and evaluated student work for undergraduate students in
``\A{ARCH} \LN{1000}: Introduction to Careers in Design and Construction''
(Instructor: Prof. Tarik Orgen).

\suppresstrue

\section*{Earlier Experience} % --------------------------------------------- %

\AdjSectSpace
\suppressfalse

\subsection*{%
  The Freelon Group\DotSep{0.25em} Durham, NC\DotSep{0.25em}%
  \space\textit{(joined Perkins+Will in March 2014)}%
}

\subsubsection*{Intern\DotSep{0.25em} June -- August 2012}

Produced schematic design drawings, digital models (Rhino, SketchUp, Revit),
presentation boards (Illustrator, InDesign) for multiple renovation projects
and proposals, including: Martin Luther King Jr. Memorial Library (Washington,
DC); North Carolina Museum of History (Raleigh, NC); John Avery Boys and Girls
Club (Durham, NC).


\subsubsection*{Intern\DotSep{0.25em} May -- August 2010}

Created schematic design drawings, digital models (Rhino, SketchUp), physical
models, and presentation boards (Illustrator, InDesign) for several
higher-education and museum projects and design bids.

Produced presentation slides, diagrams (Illustrator), and videos (Adobe
Premiere) documenting the office’s portfolio of design work, including the
Smithsonian National Museum of African American History and Culture
(Washington, DC), the Harvey B. Gantt Center (Charlotte, NC), and
other projects.

Worked directly with the company president (Phil Freelon, \A{FAIA}) on
pre-design precedent studies and drawings for the Freelon > \A{REACH}
exhibition (Wolk Gallery, \A{MIT} Department of Architecture; Cambridge, MA)
and accompanying booklet, as well as design studies for a limited-edition
coffee table (manufactured by Herman Miller).


\subsection*{PEC Structural Engineering, Inc.\DotSep{0.25em} Huntsville, AL}

\subsubsection*{Intern\DotSep{0.25em} May -- August 2007}

Updated `red-lined' revisions to construction drawings (Auto\A{CAD}), and
assisted in on-site inspections and documentation of various municipal and
single-family residential projects.

\subsection*{Bentley Systems, Inc.\DotSep{0.25em} Madison, AL}

\subsubsection*{Intern\DotSep{0.25em} May -- August 2006}

Conducted tests on eWarehouse, an application for plant operators to manage
maintenance of industrial equipment. Reported and tracked trouble reports and
change requests via database (FlawTrack). Created MS Excel macros to organize
trouble report and change request data.



\section*{Awards + Recognitions + Affiliations} % --------------------------- %

\Topic{Student category finalist (2014)}\DotSep{0.25em} Fast Company Innovation
by Design Awards (in collaboration with \A{MIT} Media Laboratory Changing
Places Group team)

\Topic{Robert R. Taylor Fellowship (2012-14)}\DotSep{0.25em} \A{MIT} School of
Architecture + Planning \A{(SA+P)}

\Topic{Honorable Mention (2012)}\DotSep{0.25em} Pella Design Portfolio
Competition, Auburn University School of Architecture, Planning, and Landscape
Architecture \A{(APLA)}

\Topic{1\Ord{st} place (2011)}\DotSep{0.25em} Student Design Competition,
National Organization of Minority Architects (in collaboration with Auburn
University \A{NOMAS} competition team)

\Topic{1\Ord{st} place (2011)}\DotSep{0.25em} Blackwell Prize in Drawing \&
Painting, Auburn University \A{APLA}

\Topic{Faculty \& Staff Award (2011)}\DotSep{0.25em} Auburn University \A{APLA}

\Topic{1\Ord{st} place (2011)}\DotSep{0.25em} Pella Design Portfolio
Competition, Auburn University \A{APLA}

\Topic{1\Ord{st} place (2010)}\DotSep{0.25em} Architecture Writing Award, Auburn
University \A{APLA}

\Topic{Cooper Carry Architects Annual Scholarship (2009)}\DotSep{0.25em}
Auburn University \A{APLA}

\Topic{Dean's List}\DotSep{0.25em} Spring 2008,\enspace Summer 2008,\enspace
Fall 2010\DotSep{0.25em} Auburn University College of Architecture, Design and
Construction \A{(CADC)}

\hfill
\vspace{-15pt}

\Topic{President (2011-12)}\DotSep{0.25em} National Organization of Minority
Architecture Students \A{(NOMAS)}, Auburn University chapter

\Topic{5\Ord{th}-Year Student Representative (2011-12)}\DotSep{0.25em}
American Institute of Architecture Students \A{(AIAS)}, Auburn University
chapter

\Topic{Member} \DotSep{0.25em} Golden Key International Honour Society, Auburn
University chapter; inducted Spring 2009

\Topic{Member} \DotSep{0.25em} National Society of Collegiate Scholars, Auburn
University chapter; inducted Spring 2009



\section*{Publications} % --------------------------------------------------- %

\hangindent=10pt
Haymaker, J., Bernal, M., Marshall, M. T., Okhoya, V., Szilasi, A., Rezaree,
R., Chen, C., Salveson, A., Brechtel, J., Hasan, H., Ewing, P. H., and Welle,
B.\LN{(2018)}. ``Design Space Construction: A Framework to Support
Collaborative, Parametric Decision Making.'' \ProperPub{Journal of Information
Technology in Construction (ITCon)}, Vol. 23, pp. 157-178. \A{URL:}
\textit{\url{https://www.itcon.org/paper/2018/8}}

\hangindent=10pt
Ewing, P. H., Haymaker, J., and Edelstein, E. A.\LN{(2017)}. ``Simulating
Circadian Light: Multi-Dimensional Illuminance Analysis.'' Presented at
International Building Performance Simulation Association \A{(IBPSA)} Building
Simulation 2017 conference, San Francisco, CA, \A{USA}, August 7-9, 2017.
\A{URL:} \textit{\url{http://www.ibpsa.org/proceedings/BS2017/BS2017_660.pdf}}

\hangindent=10pt
Ewing, P. H.\LN{(2015)}. ``Interactive Phototherapy: Integrating Photomedicine
Into Interactive Architecture.'' Thesis: S.M., Massachusetts Institute of
Technology \A{(MIT)}, Department of Architecture. Advisor: Kent Larson.
\A{URL:} \textit{\url{http://hdl.handle.net/1721.1/99275}}

\section*{Supplementary Links} % -------------------------------------------- %

\Topic{Personal Website}\DotSep{0.25em}
\textit{\url{http://www.phillipewing.com}}

\Topic{LinkedIn Profile}\DotSep{0.25em}
\textit{\url{http://www.linkedin.com/in/phewing}}

\Topic{CityHome Video}\DotSep{0.25em}
\textit{\url{https://www.youtube.com/watch?v=f8giE7i7CAE}}

\vfill

\begin{center}

  \textit{\LaTeXe\ source code for CV available on GitHub:}

  \textit{\url{https://github.com/phewing/phewing-CV}}

\end{center}



\end{document}