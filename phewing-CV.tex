% Author: Phillip Hampton Ewing, Jr.
% Email: phewing@alum.mit.edu
% Website: www.phillipewing.com

% --------------------------------------------------------------------------- %

\documentclass[letterpaper, oneside, 10pt]{article}

\usepackage{multicol}
\usepackage{hologo}
\usepackage{phewing-cv}

% --------------------------------------------------------------------------- %

\renewcommand{\FirstName}{Phillip}
\renewcommand{\MiddleName}{Hampton}
\renewcommand{\LastName}{Ewing}
\renewcommand{\NameSuffix}{Jr.}
\renewcommand{\PhoneNumber}{\input{phone-number.txt}}
\renewcommand{\Email}{phewing@alum.mit.edu}
\renewcommand{\Website}{www.phillipewing.com}

\setcounter{secnumdepth}{0}
\renewcommand{\PageCount}{6}

% =========================================================================== %

\begin{document}
\StartCV%

\section*{Summary} % -------------------------------------------------------- %

Cross-disciplinary designer, researcher, analyst, developer, and educator
exploring innovative solutions for the built environment---\textit{and} the
objects, processes, systems, and technologies which occupy and define it.
Expanding upon a training in architectural design, professional experience
includes (but is not limited to) \LN{5+} years of work on:

\vspace{-1em}

\begin{itemize}[label={--}, itemsep=0em, leftmargin=1em]

  \item Formulating parametric, generative, and simulation-based software tools
  for solving design problems;

  \item Conducting field measurements, occupant surveys, and pilot studies of
  photometric, thermal, and acoustical conditions (and associated
  psychometrics) in interior spaces;

  \item Integrated hardware production, including evaluating component
  datasheets, \A{PCB} layout and population, various digital fabrication
  processes, embedded programming and debugging, assembly integration, optical
  and servomotor calibration, and testing;

  \item Translating research in neuroscience, bioinformatics, artificial
  intelligence, atmospheric science, computer graphics, and other disciplines
  into insights relevant to designers; and,

  \item Communicating building technology topics and techniques to students,
  professionals, and researchers via peer-reviewed research, course curricula,
  lectures, and workshops.

\end{itemize}



\section*{Education} % ------------------------------------------------------ %
\AdjSectSpace

\subsection*{%
  Massachusetts Institute of Technology (MIT)%
  \DotSep{0.25em} Cambridge, MA%
}
  \subsubsection*{%
    Master of Science In Architecture Studies (SMArchS),%
    \ Design and Computation\DotSep{0.25em} 2015%
  }
    \TopicEntry{Thesis}\ProperPub{%
      Interactive Phototherapy: Integrating Photomedicine into\ %
      Interactive Architecture%
    }\\%
    \TopicEntry{Cumulative GPA}\LN{4.8 (of 5.0)}%


\subsection*{Auburn University\DotSep{0.25em} Auburn, AL}
  \subsubsection*{Bachelor of Architecture (BArch)\DotSep{0.25em} 2012}
  \vspace{-6.5pt}
  \subsubsection*{Bachelor of Interior Architecture (BIArch)\DotSep{0.25em} 2012}
    \TopicEntry{Honors}\Foreign{Magna Cum Laude}\\
    \TopicEntry{Cumulative GPA}\LN{3.6 (of 4.0)}


\subsection*{%
  University of Alabama in Huntsville (UAH)%
  \DotSep{0.25em} Huntsville, AL%
}
  \subsubsection*{(General Studies)\DotSep{0.25em} 2006-2007}
    \TopicEntry{Cumulative GPA}\LN{4.0 (of 4.0)}



\section*{Skills} % --------------------------------------------------------- %
\AdjSectSpace%

\setlength{\multicolsep}{10pt}
\begin{multicols}{2}
{%
  \RaggedRight%
  {\jostmedium 3D/CAD/CAE:}%
    \begin{itemize*}[%
      label=\relax, labelwidth=0pt, itemjoin=\space\char"00B7%
    ]%
      \item \LN{3ds Max}%
      \item Auto\A{CAD}%
      \item Blender%
      \item Digital Project%
      \item Dynamo%
      \item \A{EAGLE}%
      \item Grasshopper%
      \item Inventor%
      \item Maya%
      \item Revit%
      \item \LN{Rhino 3D}%
      \item SketchUp%
      \item SolidWorks%
    \end{itemize*}


  {\jostmedium Analog Fabrication:}%
    \begin{itemize*}[%
      label=\relax, labelwidth=0pt, itemjoin=\space\char"00B7%
    ]%
      \item Composites molding/casting%
      \item Microelectronics production%
      \item Model-making/prototyping%
      \item Plastics molding/casting%
    \end{itemize*}


  {\jostmedium Building Simulation:}%
    \begin{itemize*}[%
      label=\relax, labelwidth=0pt, itemjoin=\space\char"00B7%
    ]%
      \item \A{ALFA}%
      \item \A{DIVA}%
      \item Ecotect%
      \item EnergyPlus%
      \item \A{IES VE}%
      \item OpenStudio%
      \item Radiance%
    \end{itemize*}


  {\jostmedium CAM:}%
    \begin{itemize*}[%
      label=\relax, labelwidth=0pt, itemjoin=\space\char"00B7%
    ]%
      \item \A{ABB} Robot Studio%
      \item \A{CNC}js%
      \item Fab Modules \A{(MIT)}%
      \item \LN{PartWorks 2D}%
      \item Surf\A{CAM}%
    \end{itemize*}


  {\jostmedium Digital Fabrication:}%
    \begin{itemize*}[%
      label=\relax, labelwidth=0pt, itemjoin=\space\char"00B7%
    ]%
      \item \LN{3D printing}%
      \item \A{CNC} milling%
      \item Laser cutting%
      \item Vinyl cutting%
      \item Waterjet cutting%
    \end{itemize*}


  {\jostmedium Libraries/Frameworks:}%
    \begin{itemize*}[%
      label=\relax, labelwidth=0pt, itemjoin=\space\char"00B7%
    ]%
      \item Arduino \A{SDK} (C/C++)%
      \item Google Maps \A{API} (JavaScript)%
      \item Grasshopper \A{SDK} (C\#, IronPython)%
      \item OpenCV (C++, Python)%
      \item TensorFlow w/Keras (Python)%
      \item Three.js (JavaScript)%
    \end{itemize*}


  {\jostmedium Markup:}%
    \begin{itemize*}[%
      label=\relax, labelwidth=0pt, itemjoin=\space\char"00B7%
    ]%
      \item \A{CSS}%
      \item \A{HTML}%
      \item Markdown%
    \end{itemize*}


  {\jostmedium Media Creation:}%
    \begin{itemize*}[%
      label=\relax, labelwidth=0pt, itemjoin=\space\char"00B7%
    ]%
      \item Dreamweaver%
      \item Final Cut Pro%
      \item Illustrator%
      \item InDesign%
      \item Inkscape%
      \item Photoshop%
      \item Premeire%
      \item Scribus%
    \end{itemize*}


  {\jostmedium Programming/Scripting:}%
    \begin{itemize*}[%
      label=\relax, labelwidth=0pt, itemjoin=\space\char"00B7%
    ]%
      \item C/C++%
      \item C\#%
      \item Java%
      \item JavaScript%
      \item \LaTeX\ /\ \hologo{BibTeX}%
      \item Python%
      \item R%
      \item \A{SQL}
    \end{itemize*}

  {\jostmedium (Miscellaneous):}%
    \begin{itemize*}[%
      label=\relax, labelwidth=0pt, itemjoin=\space\char"00B7%
    ]%
      \item bash%
      \item CMake%
      \item \A{DOT}%
      \item Git%
      \item \A{HDF5}%
      \item libradtran%
      \item Mercurial%
      \item \A{MS-DOS}%
      \item Visual Paradigm (\A{UML}/SysML)%
    \end{itemize*}
  }
\end{multicols}

\section*{Recent Experience} % ---------------------------------------------- %

\hfill
\vspace{-24pt}

\subsection*{Mango (Dx) Inc.\DotSep{0.25em} Huntsville, AL}
\subsubsection*{%
  Systems Analyst (Contractor)\DotSep{0.25em} January 2023 --- March 2023
}

\textit{(Abstract)}\DotSep{0.25em} Research and development of (a) a novel
sub-pixel lensless microscopy system, (b) for antimicrobial susceptibility and
bioburden testing, (c) for large-scale industrial pharmaceutical applications.

Analyzed and proposed improvements for prototype hardware design (particularly
components and assemblies requiring additive manufacturing/\LN{3D } printing
processes) by reviewing parametric design models (SolidWorks) and test-fitting,
modification, and assembly of physical prototypes.

Evaluated preliminary super-resolution reconstructed image results of microbial
specimens, and proposed (potential) solutions for both system hardware design
and software algorithm specifications.

Managed all aspects of hardware lab inventory and equipment, and evaluated
suitability of prototype parts received from third-party vendors.


\subsection*{iCubate (MDx) Inc.\DotSep{0.25em} Huntsville, AL}
\subsubsection*{Systems Analyst\DotSep{0.25em} May 2020 --- September 2021}

\textit{(Abstract)}\DotSep{0.25em} Production of (a) a novel (US Patent No.
10,345,320), multiplexed, arm-\A{PCR} MDx platform, (b) consisting of
``iC-Processor'' machine for running automated biological assays, (c) on
biological samples in a self-contained ``iC-Cassette'', (d) after which the
results may be read by an ``iC-Reader'' machine, and (e) received by a desktop
computer configured with ``iC-Report'' software for analyzing, displaying and
storing assay results.

Oversaw and streamlined daily operations and increased system production rates
for the hardware manufacturing lab, (a) per agendas perscribed by the (\A{CTO}
and \A{COO}), while (b) enforcing compliance with iCubate's \A{FDA}-cleared
quality system (QS) protocols regarding system production.

Performed specialized assembly, alignment, and/or calibration of
micron-precision laser/optical and servo-mechanical sub-assemblies.

Documented any/all inventory transactions of hardware components and
work-in-progress assemblies, or other materials.

Trained hardware technicians on procedures for hardware sub-assemblies,
assembly inspections, testing, and related procedures.

Assigned assembly tasks and their priority levels to hardware technician
individuals/teams to ensure product delivery deadlines.

Performed troubleshooting, diagnostics, and repairs on any malfunctioning
production devices (both internal-use and customer-facing).

Verified all precision manufacturing test equipment received regular
calibrations from a third-party vendor.

Developed scripts (Python) to extract and reformat pre-existing (\A{PDF }
format) report data to aid in ongoing (internal) assay validation studies.

Assisted Senior Software Engineer on \A{SQL} database scripts for uploading
special/preliminary sets of cassette data to a business management system
(\A{BMS}; part of ``iC-Report'' software).

Advised quality system administrators (QS/QA) on any discrepancies, process
changes, or other concerns encountered during hardware assembly runs or
software validation, for revision to Standard Operating Procedures (\A{SOP}s)
and subsidiary protocol documents.



\subsection*{Perkins+Will\DotSep{0.25em} Atlanta, GA}
\subsubsection*{Researcher\DotSep{0.25em} May 2016 -- June 2019}


\textit{(Abstract)}\DotSep{0.25em} Initiated and/or collaborated with other
researchers in a broad variety of topics advancing architectural technologies
and design techniques. Further, responsibilities included advising/supporting
specific project teams, especially by implementing ``Design Space
Construction'' as a design workflow to (a) clarify architectural design
requirements, (b) identify key performance indicators, (c) perform lighting and
energy simulations on all (or selected) possible design permutations, in order
to (d) evaluate effects of specific design parameters, thus leading to (e)
data-informed decision-making and enhanced project performance overall.

Directly reported to the Director of Research for overall work, while also
collaborating with Directors of ``Neuro-architecture'', ``Energy'' [Analysis],
[Design] ``Process'', and "Building Technology" Labs for research, development,
and/or consulting work on specific projects. An abridged listing of projects
and activities, grouped according to role, is included in subsequent entries:

\textit{Primary Applied Research}\DotSep{0.25em} (a) Unilateral software and
hardware development of a wearable circadian light tracker (funded by an
in-house research micro-grant); development of a tool for multi-dimension
circadian light analysis; (b) front-end and mid-level software development
(e.g. JavaScript, Three.js, shell-scripts) for ``Simulation Platform for Energy
Efficient Design (SPEED)'' an internal software platform for design teams to
perform mass-scale (e.g., 1000+ iterations in parallel) conceptual energy
analysis simulations and analyses in "the cloud"; (c) continuing development of
Design Space Construction (\A{DSC}), a software tool/workflow (Rhino
|LN{3D}/Grasshopper \& assorted plugins) allowing for data-informed decision
making in the design process.

\textit{Secondary Applied Research}\DotSep{0.25em} (a) Internal demonstrations
and pilot studies (e.g. field measurements, data analysis) regarding techniques
for evaluating architectural psychoacoustics and thermal comfort in
pre-existing and completed client projects; (b) experimentation (hosted at
Autodesk \A{BUILD} Space in Boston, MA) with developing techniques for
architects to deploy processes involving simultaneous operation of multiple
coordinated robotic arms for novel construction techniques (thermoplastic
robotics), using previous work on thermoplastic latticework as a test case; (c)
evaluation and advise of other researchers' work regarding development of an
automated space plan generator (SPG) given programmatic requirements; (d)
supervision (for the latter phase of) an ongoing study in the Atlanta, GA
Perkins+Will office pertaining to the impact standing desks in the workplace
(``Stand Up to Work'').

\textit{Education}\DotSep{0.25em} (a) Leading and/or contributing to workshops
teaching designers how to apply research lab processes and tools such as design
space construction, circadian light analysis, and other research lab
developments to client projects in multiple Perkins+Will offices (e.g., Boston,
MA; Seattle, WA; San Francisco, CA); (b) delivering workshops and/or lectures
multiple universities (University of Washington in Seattle; University of
Oregon, Georgia Institute of Technology; Auburn University) on aforementioned
subjects.

\textit{Equipment Acquisition}\DotSep{0.25em} Responsible for management and
training other research lab colleagues on operation of sound level and noise
dose meters, handheld spectrometers, a distributed ambient temperature and
relative humidity sensor system prototype (``Pointelist''), and a smart
flooring system, among other instruments.

\suppresstrue


\section*{Teaching Experience} % -------------------------------------------- %

\hfill
\AdjSectSpace

\subsection*{Auburn University\DotSep{0.25em} Auburn, AL}


\subsubsection*{Guest Lecturer\DotSep{0.25em} August -- December 2018}

Developed and presented three lectures, three workshops, and assignments on
building performance analysis using the Design Space Construction workflow
(Grasshopper, EnergyPlus, Radiance) to the entire 4\Ord{th}-year architecture
student body in ``\A{ARCH} \LN{2220}: Environmental Controls II'' (Instructor:
Prof. David Hinson, \A{FAIA}). Reviewed and graded student assignments.


\subsubsection*{Guest Lecturer + Critic\DotSep{0.25em} June – August 2018}

Presented two lectures on acoustics and lighting analysis to 4th-year
undergraduate students in ``\A{ARIA} \LN{4030}: Interior Architecture
Thesis'' studio (Instructors: Profs. Kevin Moore, Matt Hall, Lida Sease).
Conducted two desk critique sessions and participated as a juror for midterm
and final reviews.


\subsubsection*{Guest Critic\DotSep{0.25em} April 2018}

Invited and attended as a design juror to review undergraduate thesis student
projects for ``\A{ARCH} \LN{5020}: Thesis Studio'' (Instructors: Profs. Justin
Miller, Randal Vaughn).

\subsection*{University of Washington\DotSep{0.25em} Seattle, WA}

\subsubsection*{Guest Lecturer\DotSep{0.25em} October 2017}

Co-presented a lecture and on circadian light analysis (Grasshopper, Radiance,
Python) for graduate students in the seminar course ``\A{ARCH}\LN{526}:
Topics in High Performance Buildings'' (Instructors: Prof. Chris Meek, Devin
Kleiner, Heather Burpee) as part of a workshop on the Design Space
Construction \A{(DSC)} workflow for building performance analysis. Co-wrote
the documentation and instructions on running \A{(DSC)} workflow.

\setlength{\pagetotal}{\pagetotal - 3em}

\subsubsection*{Guest Lecturer\DotSep{0.25em} October 2017}

Co-presented a lecture on circadian light analysis (Grasshopper, Radiance,
Python) and conducted desk critiques of projects for graduate architecture
students enrolled in ``\A{ARCH} \LN{504}: Graduate Design Studio''
(Instructors: Prof. Chris Meek, David Kleiner).

\subsection*{%
  Singapore University of Technology and Design (SUTD)%
  \DotSep{0.25em} Singapore, SG\DotSep{0.25em}\textit{and}%
}
\subsection*{%
  Massachusetts Institute of Technology (MIT)\DotSep{0.25em} Cambridge, MA%
}

\subsubsection*{Research Assistant\DotSep{0.25em} June -- August 2013}

Developed a curriculum for an introductory class on Building Information
Modeling \A{(BIM)} for the Singapore University of Technology and Design
(\A{SUTD}; in collaboration as a `sister' university of \A{MIT}), emphasizing
\A{BIM} as part of a larger conceptual framework incorporating digital
fabrication workflows and ``digital twins'' for building operation and
monitoring. Produced semester-long course syllabus, content plan, and the
first five weeks of: lecture outlines, step-by-step lab session instructions,
digital models (Revit), and student assignments.

\suppressfalse

\subsection*{Auburn University\DotSep{0.25em} Auburn, AL}

\subsubsection*{Student Instructor\DotSep{0.25em} August 2011 -- May 2012}

Developed teaching materials and instructed workshops on Rhino, Grasshopper,
and design application interoperability for undergraduate and graduate
architecture students. Advised on the development of additional digital media
workshops.

\subsection*{Auburn University\DotSep{0.25em} Auburn, AL}

\subsubsection*{Teaching Assistant\DotSep{0.25em} August -- December 2010}

Led lab sessions and evaluated student work for undergraduate students in
``\A{ARCH} \LN{1000}: Introduction to Careers in Design and Construction''
(Instructor: Prof. Tarik Orgen).


\section*{Earlier Experience} % --------------------------------------------- %

\AdjSectSpace
\suppressfalse

\subsection*{%
  The Freelon Group\DotSep{0.25em} Durham, NC\DotSep{0.25em}%
  \space\textit{(joined Perkins+Will in March 2014)}%
}

\subsubsection*{Intern\DotSep{0.25em} June -- August 2012}

Produced schematic design drawings, digital models (Rhino, SketchUp, Revit),
presentation boards (Illustrator, InDesign) for multiple renovation projects
and proposals, including: Martin Luther King Jr. Memorial Library (Washington,
DC); North Carolina Museum of History (Raleigh, NC); John Avery Boys and Girls
Club (Durham, NC).


\subsubsection*{Intern\DotSep{0.25em} May -- August 2010}

Created schematic design drawings, digital models (Rhino, SketchUp), physical
models, and presentation boards (Illustrator, InDesign) for several
higher-education and museum projects and design bids.

Produced presentation slides, diagrams (Illustrator), and videos (Adobe
Premiere) documenting the office’s portfolio of design work, including the
Smithsonian National Museum of African American History and Culture
(Washington, DC), the Harvey B. Gantt Center (Charlotte, NC), and
other projects.

Worked directly with the company president on pre-design precedent studies and
drawings for the Freelon > \A{REACH} exhibition (Wolk Gallery, \A{MIT}
Department of Architecture; Cambridge, MA) and accompanying booklet, as well as
design studies for a limited-edition coffee table (manufactured by Herman
Miller).


\subsection*{PEC Structural Engineering, Inc.\DotSep{0.25em} Huntsville, AL}

\subsubsection*{Intern\DotSep{0.25em} May -- August 2007}

Updated `red-lined' revisions to construction drawings (Auto\A{CAD}), and
assisted in on-site inspections and documentation of various municipal and
single-family residential projects.

\subsection*{Bentley Systems, Inc.\DotSep{0.25em} Madison, AL}

\subsubsection*{Intern\DotSep{0.25em} May -- August 2006}

Conducted tests on eWarehouse, an application for plant operators to manage
maintenance of industrial equipment. Reported and tracked trouble reports and
change requests via database (FlawTrack). Created MS Excel macros to organize
trouble report and change request data.


\section*{Awards + Recognitions + Affiliations} % --------------------------- %

\Topic{3\Ord{rd} Place, College/Professional Category (2020)}\DotSep{0.25em}
HudsonAlpha Tech Challenge Hackathon (in collaboration with ``The Missing
Links'' project team)

\Topic{Board member (2019 -- Present)}\DotSep{0.25em} Interior Architecture
Program Advisory Council, Auburn University School of Architecture, Planning,
and Landscape Architecture \A{(APLA)} 

\Topic{Student category finalist (2014)}\DotSep{0.25em} Fast Company Innovation
by Design Awards (in collaboration with \A{MIT} Media Laboratory Changing
Places Group team)

\Topic{Robert R. Taylor Fellowship (2012-14)}\DotSep{0.25em} \A{MIT} School of
Architecture + Planning \A{(SA+P)}

\Topic{Honorable Mention (2012)}\DotSep{0.25em} Pella Design Portfolio
Competition, Auburn University \A{(APLA)}

\Topic{1\Ord{st} place (2011)}\DotSep{0.25em} Student Design Competition,
National Organization of Minority Architects (in collaboration with Auburn
University \A{NOMAS} competition team)

\hfill
\vspace{1.5pt}

\Topic{1\Ord{st} place (2011)}\DotSep{0.25em} Blackwell Prize in Drawing \&
Painting, Auburn University \A{APLA}

\Topic{Faculty \& Staff Award (2011)}\DotSep{0.25em} Auburn University \A{APLA}

\Topic{1\Ord{st} place (2011)}\DotSep{0.25em} Pella Design Portfolio
Competition, Auburn University \A{APLA}

\Topic{1\Ord{st} place (2010)}\DotSep{0.25em} Architecture Writing Award, Auburn
University \A{APLA}

\Topic{Cooper Carry Architects Annual Scholarship (2009)}\DotSep{0.25em}
Auburn University \A{APLA}

\Topic{Dean's List}\DotSep{0.25em} Spring 2008,\enspace Summer 2008,\enspace
Fall 2010\DotSep{0.25em} Auburn University College of Architecture, Design and
Construction \A{(CADC)}

\Topic{President (2011-12)}\DotSep{0.25em} National Organization of Minority
Architecture Students \A{(NOMAS)}, Auburn University chapter

\Topic{5\Ord{th}-Year Student Representative (2011-12)}\DotSep{0.25em}
American Institute of Architecture Students \A{(AIAS)}, Auburn University
chapter

\Topic{Member (ind. Spring 2009)}\DotSep{0.25em} Golden Key International
Honour Society, Auburn University chapter

\Topic{Member (ind. Spring 2009)}\DotSep{0.25em} National Society of
Collegiate Scholars, Auburn University chapter



\section*{Publications} % --------------------------------------------------- %

\hangindent=10pt
Haymaker, J., Bernal, M., Marshall, M. T., Okhoya, V., Szilasi, A., Rezaree,
R., Chen, C., Salveson, A., Brechtel, J., Hasan, H., Ewing, P. H., and Welle,
B.\LN{(2018)}. ``Design Space Construction: A Framework to Support
Collaborative, Parametric Decision Making.'' \ProperPub{Journal of Information
Technology in Construction (ITCon)}, Vol. 23, pp. 157-178. \A{URL:}
\textit{\url{https://www.itcon.org/paper/2018/8}}

\hangindent=10pt
Ewing, P. H., Haymaker, J., and Edelstein, E. A.\LN{(2017)}. ``Simulating
Circadian Light: Multi-Dimensional Illuminance Analysis.'' Presented at
International Building Performance Simulation Association \A{(IBPSA)} Building
Simulation 2017 conference, San Francisco, CA, \A{USA}, August 7-9, 2017.
\A{URL:} \textit{\url{http://www.ibpsa.org/proceedings/BS2017/BS2017_660.pdf}}

\hangindent=10pt
Ewing, P. H.\LN{(2015)}. ``Interactive Phototherapy: Integrating Photomedicine
Into Interactive Architecture.'' Thesis: S.M., Massachusetts Institute of
Technology \A{(MIT)}, Department of Architecture. Advisor: Kent Larson.
\A{URL:} \textit{\url{http://hdl.handle.net/1721.1/99275}}

\suppresstrue

\section*{Supplementary Links} % -------------------------------------------- %

\Topic{Personal Website}\DotSep{0.25em}
\textit{\url{http://www.phillipewing.com}}

\Topic{LinkedIn Profile}\DotSep{0.25em}
\textit{\url{http://www.linkedin.com/in/phewing}}

\Topic{CityHome Video}\DotSep{0.25em}
\textit{\url{https://www.youtube.com/watch?v=f8giE7i7CAE}}

\vfill

\begin{center}

  \textit{\LaTeXe\ source code for CV available on GitHub:}

  \textit{\url{https://github.com/phewing/phewing-CV}}

\end{center}



\end{document}